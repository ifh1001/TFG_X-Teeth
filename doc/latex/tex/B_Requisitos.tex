\apendice{Especificación de Requisitos}

\section{Introducción}
En este apartado se habla acerca de los diferentes objetivos generales del proyecto, junto con los diferentes requisitos y casos de uso.

\section{Objetivos generales}
Los objetivos generales del proyecto son:
\begin{itemize}
    \item Investigar y aplicar nuevas funcionalidades del \emph{deep learning} en la rama de la odontología.
    \item Desarrollar una aplicación web, sencilla de usar y con un fácil acceso para todo el mundo.
    \item Ayudar en el proceso de la endodoncia, de forma que la obtención de la longitud, actualmente hecha de forma manual, se pueda automatizar con la aplicación.
    \item Intentar conseguir que el error de la aplicación en la estimación de la longitud del diente no sea superior a 0.5 milímetros.
\end{itemize}

\section{Catalogo de requisitos}
En esta sección se hablará acerca de los distintos requisitos que tiene la aplicación final. Adicionalmente, se indicará quien es el actor de dicha aplicación.

Actor de la aplicación:
\begin{description}
\item[Odontólogo:] Será la persona que una vez haya realizado la radiografía del diente del paciente, la introduzca en la aplicación para obtener los resultados de la longitud de dicho diente.
\end{description}

Los requisitos funcionales de la aplicación son:
\begin{itemize}
	\item \textbf{REQ.1:} Llevar a cabo una aplicación capaz de calcular la longitud del diente de una radiografía.
	\begin{itemize}
	\item \textbf{REQ.1.1:} El odontólogo podrá cargar la/las radiografías deseadas.
	\item \textbf{REQ.1.2:} El odontólogo podrá ejecutar tantas veces como quiera la aplicación con las radiografías que desee.
	\item \textbf{REQ.1.3:} El odontólogo podrá ver que las imágenes se han subido correctamente.
	\end{itemize}
	
	\item \textbf{REQ.2:} Llevar a cabo una aplicación capaz de mostrar la segmentación de la radiografía junto con la recta de cálculo de distancia.
	\begin{itemize}
	\item \textbf{REQ.2.1:} El odontólogo podrá ver la segmentación del diente y del nervio.
	\item \textbf{REQ.2.2:} El odontólogo podrá ver la recta sobre la que se obtiene la longitud del diente.
	\item \textbf{REQ.2.3:} El odontólogo podrá ver adicionalmente la radiografía original.
	\end{itemize}
\end{itemize}

\newpage
\section{Especificación de requisitos}
\subsection{Diagrama de casos de uso}
\imagen{TFG-Casos de Uso.drawio}{Diagrama de Casos de Uso de la Aplicación}

