\capitulo{2}{Objetivos del proyecto}

Este apartado contiene los objetivos del proyecto. Se pueden distinguir dos tipos: generales y técnicos. Dichos objetivos buscan cumplirse con el desarrollo del propio proyecto.

\section{Objetivos Generales}
Los objetivos generales del proyecto son:
\begin{itemize}
    \item Investigar y aplicar nuevas funcionalidades del \emph{deep learning} en la rama de la odontología.
    \item Desarrollar una aplicación web, sencilla de usar y con un fácil acceso para todo el mundo.
    \item Ayudar en el proceso de la endodoncia, de forma que la obtención de la longitud, actualmente hecha de forma manual, se pueda automatizar con la aplicación.
    \item Intentar conseguir que el error de la aplicación en la estimación de la longitud del diente no sea superior a 0.5 milímetros.
\end{itemize}

\section{Objetivos Técnicos}
Los objetivos técnicos del proyecto son:
\begin{itemize}
    \item Crear una aplicación web capaz de calcular la longitud de un diente a través de una radiografía.
    \item Permitir el uso de la aplicación de dos modos distintos:
    \begin{itemize}
        \item Acceder directamente al \emph{notebook} en \emph{Google Colab}.
        \item Usar \emph{ngrok} para acceder a la dirección web que contiene el \emph{notebook} con la aplicación en Gamma.
    \end{itemize}
    \item Usar el lenguaje de programación \emph{Python} para realizar las pruebas pertinentes y la aplicación final, junto con las distintas librerías que existen en \emph{Python}.
    \item Conseguir que los tiempos de espera en la aplicación sean los más breves posibles.
    \item Utilizar la plataforma de \emph{GitHub}, de forma que se vea reflejado el trabajo incremental que se lleva a cabo en cada \emph{sprint}.
\end{itemize}

\section{Objetivos Personales}
Los objetivos marcados a nivel personal son:
\begin{itemize}
    \item Aprender sobre el proceso de la endodoncia, para poder investigar y desarrollar una aplicación que permita ayudar a los odontólogos en su trabajo.
    \item Aplicar los conocimientos adquiridos a lo largo del grado universitario.
    \item Mejorar los conocimientos de \emph{Python} y aplicar aquellos ya adquiridos.
    \item Aprender sobre el \emph{deep learning} para conocer sus posibilidades y poder aplicarlas a lo largo del proyecto.
\end{itemize}